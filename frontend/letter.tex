% ============================================
% IMPORTANT: You MUST use XeLaTeX, NOT pdflatex!
% ============================================
% Compile with: xelatex letter.tex
% 
% If you get fontspec error, it means you're using pdflatex
% Solution: Use xelatex instead
% ============================================

\documentclass[12pt,a4paper]{article}

% Required for Arabic - only works with XeLaTeX or LuaLaTeX
\usepackage{fontspec}
\usepackage{polyglossia}
\setdefaultlanguage{arabic}
\setotherlanguage{english}

% Set Arabic font - larger and clearer
\setmainfont[Script=Arabic,Scale=1.1]{Amiri}
% If Amiri doesn't work, uncomment one of these:
% \setmainfont[Script=Arabic,Scale=1.1]{Arabic Typesetting}
% \setmainfont[Script=Arabic,Scale=1.1]{Traditional Arabic}
% \setmainfont[Script=Arabic,Scale=1.1]{Arial Unicode MS}
% \setmainfont[Script=Arabic,Scale=1.1]{Tahoma}

% Page setup - optimized margins to fit on one page
\usepackage{geometry}
\geometry{margin=1.8cm}

% For proper spacing - optimized
\setlength{\parindent}{0pt}
\setlength{\parskip}{0.3em}

% Improve text clarity
\usepackage{microtype}

% Enable RTL (Right-to-Left) for Arabic
\setRTL

\begin{document}

\begin{center}
\textbf{\large إلى إدارة كلية العلوم بالمنستير}
\end{center}

\begin{center}
\textbf{الموضوع: مطلب استجواب وتظلّم بخصوص إجراء امتحاني}
\end{center}

\vspace{0.4cm}

\noindent السلام عليكم ورحمة الله وبركاته،

\vspace{0.25cm}

\noindent يشرفني أن أتقدم إلى سيادتكم بهذا المطلب راجيًا منكم التفهّم والنظر بعين الاعتبار إلى وضعيتي.

\vspace{0.15cm}

\noindent أنا الممضي أسفله علاء ميساوي، طالب بالسنة الثانية ماجستير علوم البيانات بكلية العلوم بالمنستير.

\noindent أودّ أن أعلمكم أنّني أعمل بالتوازي مع دراستي كـ مطور برمجيات بإحدى الشركات الخاصة بسوسة، وهو ما يفرض عليّ التوفيق بين الالتزامات المهنية والدراسية في نفس الوقت، الأمر الذي سبّب لي في بعض الفترات صعوبات حقيقية في المتابعة، بل وحرمني من اجتياز بعض الامتحانات نظرًا لظروف العمل.

\vspace{0.15cm}

\noindent خلال أحد الامتحانات، وصلت متأخرًا نسبيًا إلى قاعة الامتحان، وأثناء ذلك نسيت ـ عن غير قصد ـ إغلاق هاتفي الجوال. وخلال الامتحان تلقّيت اتصالًا هاتفيًا، فصدر صوت واهتزاز من الهاتف، فقمت بعفوية بإخراجه من جيبي لإسكاته فقط، دون أي نية أو محاولة للغش.

\vspace{0.15cm}

\noindent إلا أنّ الأستاذ المراقب اعتبر ذلك مخالفة، فقام بسحب ورقة الامتحان وطلب مني إعادة كتابة الفرض في ورقة جديدة، ظنًّا منه أنني بصدد الغش، رغم أنّه لم يتم ضبط أي حالة غش فعلية، ولم يتم استعمال الهاتف لأي غرض غير إسكاته.

\vspace{0.15cm}

\noindent وأودّ التأكيد لسيادتكم أنّني طوال مسيرتي الجامعية الممتدة لخمس سنوات لم يسبق لي أبدًا الغش أو الإخلال بنظام الامتحانات، فهل يعقل أن أقترف مثل هذا الخطأ في آخر سنة من مساري الجامعي؟

\noindent إنّ ما حدث كان خطأ غير مقصود ناتجًا عن ضغط العمل والتعب الذهني، وليس تصرّفًا بنية سيئة.

\vspace{0.15cm}

\noindent كما أحيطكم علمًا أنّني بصدد استكمال إجراءات التربص ومواصلة مساري الدراسي بفرنسا، وهو ما يجعل وضعيتي الحالية حساسة وتستوجب، إن أمكن، التعجيل في النظر فيها حتى لا يتعطّل مستقبلي الأكاديمي والمهني.

\vspace{0.15cm}

\noindent لذلك، ألتمس من سيادتكم التفضّل بالنظر في طلبي هذا بعين الرحمة والإنصاف، ومراعاة ظروفي الخاصة، وتمكيني من حقي في اجتياز الامتحان دون أن أُحسب ضمن حالات الغش، إذ لم يثبت في حقي أي سلوك مخالف بنية مسبقة.

\vspace{0.15cm}

\noindent وتفضلوا، الإدارة، بقبول فائق عبارات الاحترام والتقدير.

\vspace{0.15cm}

\noindent والسلام عليكم ورحمة الله وبركاته.

\vspace{0.6cm}

\raggedright
\textbf{الإمضاء:}

علاء ميساوي

طالب ماجستير علوم البيانات

كلية العلوم بالمنستير

\end{document}
